\section{Conclusion}
\label{sec:conclusion}



In this first laboratory assignment, all the major goals of the project were achieved. We concluded with success our first interaction with a new software (Ubuntu), with a simulation platform (Ngspice), with a computational language program (GNU Octave) and with a text report editor (LaTeX). The analysis of the circuit was also finished with success through simulation and theoretical interpretation.

The main objective of the report was completed with the study of a circuit containing a voltage source $V_A$, a current-controlled voltage source $V_C$, a current source $I_D$ and a voltage-controlled current source $I_B$ connected to different fixed value resistors $R_1$, $R_2$, $R_3$, $R_4$, $R_5$, $R_6$ and $R_7$. Mesh currents and node voltage were analised both theoretically, using the Octave maths tool, and by circuit simulation, using the Ngspice tool. The simulation results matched the theoretical results precisely. This accuracy was confirmed by the matemathical calculation of relative errors, which were proved to be really small. The reason for this perfect match is the fact that this is a straightforward
circuit containing only simple and linear components, so the theoretical and simulation models cannot differ. For more complex components, the theoretical and simulation models could differ. However, this is not the case of the analysis of this report, were the results are obtained sucesfully and with notorious precision.
