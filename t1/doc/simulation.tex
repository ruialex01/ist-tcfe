\section{Simulation Analysis}
\label{sec:simulation}

\subsection{Operating Point Analysis}
\label{subsec:opa}

\begin{table}[h] \centering
  \begin{tabular}{|l|r|}
    \hline    
    {\bf Node/Component} & {\bf Value [A or V]} \\ \hline
    @cb[i] & 0.000000e+00\\ \hline
@ce[i] & 0.000000e+00\\ \hline
@q1[ib] & 7.022567e-05\\ \hline
@q1[ic] & 1.404513e-02\\ \hline
@q1[ie] & -1.41154e-02\\ \hline
@q1[is] & 5.765392e-12\\ \hline
@rc[i] & 1.411536e-02\\ \hline
@re[i] & 1.411536e-02\\ \hline
@rf[i] & 7.022567e-05\\ \hline
@rs[i] & 0.000000e+00\\ \hline
v(1) & 0.000000e+00\\ \hline
v(2) & 0.000000e+00\\ \hline
base & 2.254108e+00\\ \hline
coll & 5.765392e+00\\ \hline
emit & 1.411536e+00\\ \hline
vcc & 1.000000e+01\\ \hline

  \end{tabular}
  \caption*{Operating point. A variable preceded by @ is of type {\em current}
    and expressed in Ampere; other variables are of type {\it voltage} and expressed in
    Volts.}
  \label{tab:op}
\end{table}



\paragraph{}Ngspice is a circuit-simulation program that makes it possible to have an accurate representation of how the circuit would behave if it was actually assembled. Table~\ref{tab:op} shows the simulated operating point results for the circuit
under analysis.


\paragraph{}As it can be seen, the table shows the values of the voltage in all nodes, the currents in all of the branches and also the currents in independent voltage sources. When looking at the simulation results there is one very interesting detail which is very important: Ngspice operates with the idea that the positive current flows from the positive pole to the negative pole in all components, sources included. This explains why, for example, in the the very same branch where resistor $R_1$ and voltage source $V_A$ are, the current given by ngspice in the whole branch ($@r1[i]$) is the symetric of the one given specifically in the voltage source $V_A$. The same logic applies to $I_{Vc}$. 
\paragraph{} Another important detail about this table is the node $N_A$ and the voltage source $V_{AB} = 0 V$, which are absent from the circuit's picture. This is because the Current-Controlled Voltage Source is dependent on current $I_C$, but Ngspice requires a voltage source where this current flows through, which made necessary the use of an auxiliary voltage source in series with $R_6$ and, therefore, an auxiliary node.  



\subsection{Relative Error Analysis}

\paragraph{}
\begin{center}
\begin{tabular}{|c||c|}
      \hline    
      \multicolumn{2}{|c|} {\bf Error Table (in \% )} \\
      \hline

	
DATA & Octave & NGSpice & Percentual Relative Error \\ \hline
Output Voltage Gain & 0.990747 & 37.93151 & 3728.576821 \\ \hline
Input Impedance & 78634.698490 & 65771.74 & 16.35786585 \\ \hline
Output Impedance & 3.171596 & -3.17595 & 200.137281 \\ \hline

     
      \end{tabular}
\end{center}

\paragraph{} When the simulation results given by Ngspice are compared to the theoretical ones obtained in  Section~\ref{sec:analysis}, it is possible to highlight the fact that these are, in reality, extremely close to each other. As it can be seen, the highest error in percentual value is in the order of $10^{-4}$, which is negligible. Such result can be explained by the fact that this is a very simple circuit with very simple components, thus not having a lot of chances to differ greatly.

\paragraph{}For reasons explained in Subsection~\ref{subsec:opa}, the value considered for $I_A$ was the one given by the resistor $R_1$ and the value considered for $I_{Vc}$ was the symetric of the one given by Ngspice.

