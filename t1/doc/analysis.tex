\section{Theoretical Analysis}
\label{sec:analysis}

******************************************************************
 This circuit is carefully analysed according to two of the most efficient ways to solve a circuit: the Mesh Current Method and the Node Voltage Method.

\subsection{Mesh Current Method Analysis}

\paragraph{}
By analysing the circuit, identifying the meshes and assigning currents that work as variables, finally we can write a system of equations to help us solve the circuit through KVL:

\[
\left\{\begin{matrix}
Mesh A: R_1 \times I_a+(I_a+I_b) \times R_3+(I_a+I_c) \times R_4=V_a\\
Mesh C:	(I_a+I_c) \times R_4+R_6 \times I_c+R_7 \times I_c-K_c \times I_c=0\\
\end{matrix}\right.
\]

\paragraph{}
From here, we can rewrite the system:
\[
\left\{\begin{matrix}
Mesh A: (R_1+R_3+R_4) \times I_a+R_3 \times I_b+R_4 \times I_c=V_a\\
Mesh C: R_4 \times I_a+(R_4+R_6+R_7-K_c) \times I_c=0\\
\end{matrix}\right.
\]

\paragraph{}
It´s also important to note that $ I_b+I_5=I_d $.

\paragraph{}
From here, we can write a matrix equation that can be solved in GNU Octave:
\[
\begin{bmatrix}
R_1+R_3+R_4 & R_3 & 0 & 0 & 0 & R_4 & 0\\
R_4 & 0 & 0 & 0 & 0 & R_4+R_6+R_7-K_c & 0\\ 
1 & 0 & 0 & -1 & 0 & 1 & 0\\
\frac{1}{1-K_b \times R_3} & 0 & -1 & 0 & 0 & 0 & 0\\
0 & 0 & 0 & 0 & 0 & 1 & -1\\
0 & 1 & 0 & 0 & 1 & 0 & 0\\ 
-1 & -1 & 1 & 0 & 0 & 0 & 0\\ 
\end{bmatrix}
\begin{bmatrix}
I_a\\
I_b\\
I_3\\
I_4\\
I_5\\
I_c\\
I_{vc}\\
\end{bmatrix}
=
\begin{bmatrix}
V_a\\
0\\
0\\
0\\
I_{d}\\
I_{d}\\
0\\
\end{bmatrix}
\]

\paragraph{}
The solution to the equation is presented in the following table:
\paragraph{}
\begin{center}
\begin{tabular}{|c||c|}
      \hline    
      \multicolumn{2}{|c|} {\bf Mesh Analysis Currents} \\
      \hline

	
 Mesh Current Ia & 2.35432076795e-04 \\ \hline 
 Mesh Current Ib & -2.46391641626e-04 \\ \hline 
 Mesh Current Ic & 9.14823521600e-04 \\ \hline 
 Mesh Current Id & 1.03087754949e-03 \\ \hline 
 Branch Current Ia & 2.35432076795e-04 \\ \hline 
 Branch Current Ib & -2.46391641626e-04 \\ \hline 
 Branch Current I3 & -1.09595648304e-05 \\ \hline 
 Branch Current I4 & 1.15025559840e-03 \\ \hline 
 Branch Current I5 & 1.27726919112e-03 \\ \hline 
 Branch Current Ic & 9.14823521600e-04 \\ \hline 
 Branch Current Ivc & -1.16054027890e-04 \\ \hline 
 Branch Current Id & 1.03087754949e-03 \\ \hline 
     
      \end{tabular}
\end{center}


 

%%%%%%%%%%%%%%%%%%%%%%%%%%%%%%%%%%%%%%%%%%%%%%NODAL%%%%%%%%%%%%%%%%%%%%%%%%%%%%%%%%%%%%%%%%%%%%%%%

\subsection{Node Voltage Method Analysis}

\paragraph{}
We start to apply the Node Voltage Method by starting to identify the nodes of the circuit. In this case, our circuit has 7 nodes and each one has a node voltage designated $V_1$, $V_2$, $V_3$, $V_4$, $V_5$, $V_6$ and $V_7$, according to the related node. Then, we apply the KCL to each one of the nodes.

\[
\left\{\begin{matrix}
Node 1: I_1 + I_2 = I-3\\
Node 2: I_B=\frac{V_2-V_1}{R_2}\\
Node 3: V_3=-V_A\\
Node 4: V_4 - V_7 = V_C\\
Node 5: I_B + I_5 = I_D\\
Node 6: I_6 = I_7\\
Node 7: V_4 - V_7 = V_C\\
\end{matrix}\right.
\]

\paragraph{}
After that, we rewrite it into an equivalent system defining the currents in terms of node voltages.

\[
\left\{\begin{matrix}
Node 1: \frac{0-V_1}{R_1}+\frac{V_2-V_1}{R_2}=\frac{V_1-V_4}{R_3}\\
Node 2: I_B=\frac{V_2-V_1}{R_2}\\
Node 3: V_3=-V_A\\
Node 4: V_4 - V_7 = V_C\\
Node 5: I_B+\frac{V_5-V_4}{R_5}=I_D\\
Node 6: \frac{V_3-V_6}{R_6}=\frac{V_6-V_7}{R_7}\\
Node 7: V_4 - V_7 = V_C\\
\end{matrix}\right.
\]

\paragraph{}
Now, we have a system of 6 equations (note that applying KCL to node 4 and node 7 generate the same exact equation). Between these two nodes, the circuit presents a voltage source $V_A$. Therefore, to obtain another equation to add to the system, we need to consider a supernode that includes both node 4 and node 7 and, after that, apply KCL to the supernode we just created.

\paragraph{}
We obtain another equation: $ I_3 + I_5 + I_C = I_4 + I_D $.

\paragraph{}
Once again, we define the currents in terms of node voltages and obtain an equivalent equation: $ \frac{V_1-V_4}{R_3}+\frac{V_5-V_4}{R_5}+I_C = \frac{V_4-V_3}{R_4} + I_D $.

\paragraph{}
At this point, we have a system with 7 equations. However, besides the node voltages $V_1$, $V_2$, $V_3$, $V_4$, $V_5$, $V_6$ and $V_7$, we still have two more variables to determine its value, $I_B$ and $I_C$. So, we need to find two more equations to complete our system. We have to define the missing value currents in terms of node voltages and get the two extra equations.

\[
\left\{\begin{matrix}
I_B = \frac{V_1-V_4}{K_B} \\
I_C = \frac{V_3-V_6}{R_6} \\
\end{matrix}\right.
\]

\paragraph{}
This takes us to our final system of equations, with 9 equations to find the values of 9 variables.

\[
\left\{\begin{matrix}
V_{4} - V_{7} -K_{C}I_{C}=0\\

-V_{3}=V_{A}\\ 

-\frac{1}{R_{4}}V_{4}+\frac{1}{R_{5}}V_{5}+I_{B}=I_D\\

-\frac{1}{R_{2}}V_{1}+\frac{1}{R_{2}}V_{2}-I_{B}=0\\ 

\frac{1}{R_{6}}V_{3}+(-\frac{1}{R_{6}}-\frac{1}{R_{7}})V_{6}+\frac{1}{R_{7}}V_{7}=0\\ 

(-\frac{1}{R_{1}}-\frac{1}{R_{2}}-\frac{1}{R_{3}})V_{1} + \frac{1}{R_{2}}V_{2}+\frac{1}{R_{3}}V_{4}=0\\

K_{B}V_{1}-K_{B}V_{4}-I_{B}=0\\

\frac{1}{R_{6}}V_{3}-\frac{1}{R_{6}}V_{6}-I_{C}=0\\ 

\frac{1}{R_{3}}V_{1}+\frac{1}{R_{4}}V_{3}+(-\frac{1}{R_{3}}-\frac{1}{R_{4}}-\frac{1}{R_{5}})V_{4}+\frac{1}{R_{5}}V_{5}+I_C = I_D\\

\end{matrix}\right.
\]

\paragraph{}
We can transform this system of equations and put it into a matrix form, ready to be solved in GNU Octave.

\[
\begin{bmatrix}
0 & 0 & 0 & 1 & 0 & 0 & -1 & 0 & -K_C\\
0 & 0 & -1 & 0 & 0 & 0 & 0 & 0 & 0\\ 
0 & 0 & 0 & -\frac{1}{R_5} & \frac{1}{R_5} & 0 & 0 & 1 & 0\\
-\frac{1}{R_2} & \frac{1}{R_2} & 0 & 0 & 0 & 0 & 0 & -1 & 0\\
0 & 0 & \frac{1}{R_6} & 0 & 0 & -\frac{1}{R_6}-\frac{1}{R_7}&\frac{1}{R_7} & 0 & 0\\ 
-\frac{1}{R_1}-\frac{1}{R_2}-\frac{1}{R_3} & \frac{1}{R_2} & 0 & \frac{1}{R_3} & 0 & 0 & 0 & 0 & 0\\ 
K_B & 0 & 0 & -K_B & 0 & 0 & 0 & -1 & 0\\ 
0 & 0 & \frac{1}{R_6} & 0 & 0 & -\frac{1}{R_6} & 0 & 0 & -1\\ 
\frac{1}{R_3} & 0 & \frac{1}{R_4} & -\frac{1}{R_3}-\frac{1}{R_4}-\frac{1}{R_5} & \frac{1}{R_5} & 0 & 0 & 0 & 1
\end{bmatrix}
\begin{bmatrix}
V_1\\
V_2\\
V_3\\
V_4\\
V_5\\
V_6\\
V_7\\
I_B\\
I_C\\
\end{bmatrix}
=
\begin{bmatrix}
0\\
V_A\\
I_D\\
0\\
0\\
0\\
0\\
0\\
I_D\\
\end{bmatrix}
\]

\paragraph{}
These are the final values obtained with the appication of the Node Voltage Method:
\paragraph{}
\begin{center}
   \begin{tabular}{|c||c|}
      \hline    
      \multicolumn{2}{|c|} {\bf Nodal Analysis Voltages} \\
      \hline

        
 Node Voltage 1 & 5.02924600001e+00 \\ \hline 
 Node Voltage 2 & 5.02328431033e+00 \\ \hline 
 Node Voltage 3 & 1.69841726085e-02 \\ \hline 
 Node Voltage 5 & 5.02330236056e+00 \\ \hline 
 Node Voltage 6 & 1.27275587750e+01 \\ \hline 
 Node Voltage 7 & 0.00000000000e+00 \\ \hline 
 Node Voltage 8 & 1.57294156562e+01 \\ \hline 
     
   \end{tabular}
\end{center}   
