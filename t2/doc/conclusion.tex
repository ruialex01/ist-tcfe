\section{Conclusion}
\label{sec:conclusion}

\paragraph{}
In this first laboratory assignment, all the major goals of the project were achieved. We concluded with success a further interaction with a new software (Ubuntu), with a simulation platform (Ngspice), with a computational language program (GNU Octave) and with a text report editor (LaTeX). The analysis of the circuit was also finished with success through simulation and theoretical interpretation, which allowed a good comparative analysis between these two methods.

\paragraph{}
The main objective of the report was completed with the study of an RC circuit containing a voltage source $v_S$ (defined by the fllowing sinusoidal equation: $v_s (t) = V_s u(−t) + sin(2 \pi ft)u(t) $), a current-controlled voltage source $V_D$, a voltage-controlled current source $I_B$ and a capacitor $C$ connected to different fixed value resistors $R_1$, $R_2$, $R_3$, $R_4$, $R_5$, $R_6$ and $R_7$. Node voltages, magnitudes and phases were analised both theoretically, using the Octave maths tool, and by circuit simulation, using the Ngspice tool. The simulation results matched the theoretical results precisely. This accuracy was confirmed by the matemathical calculation of relative errors, which were proved to be really small. Also, the comparative analysis of graphics plothed by both theoretical and simulation tools confirmed the similarity of the results. The reason for this perfect match is the fact that this is a straightforward circuit containing only simple and linear components (the capacitor included), so the theoretical and simulation models cannot differ. For more complex components, the theoretical and simulation models could differ. However, this is not the case of the analysis of this report, where the results are obtained sucesfully and with notorious precision.
