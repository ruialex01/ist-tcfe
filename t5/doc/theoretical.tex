\section{Theoretical Introduction}
\label{sec:theoretical}

\paragraph{}The very fundamental concept that characterizes a Bandpass Filter is its ability to allow frequencies within a specific frequency range and reject (or attenuate) frequencies outside that range. Lowpass filters are used to isolate the signals which have frequencies higher than the cutoff frequency. Similarly, highpass filters are used to isolate the signals which have frequencies lower than the cutoff frequency. The bandpass filter is a combination of lowpass and highpass filters and is used to pass signals within a certain “band” or “spread” of frequencies without distorting the input signal. This band of frequencies can be any width and is commonly known as the filters´ bandwidth.

\paragraph{}There are many types of bandpass filter circuits. The bandpass filter circuit designed during this laboratory assignment, based upon an active bandpass filter, is a cascading connection of high pass and low pass filters with the amplifying component as shown in the figure below. The circuit diagram of this active bandpass filter is divided into three parts. The first part is for a highpass filter. Then the op-amp is used for the amplification. The last part of the circuit is the low pass filter.


\begin{figure}[H] \centering
	\includegraphics[width=0.7\linewidth]{circuit2.pdf}
	\caption{Circuit diagram of Active Bandpass Filter.}
	\label{fig:circuit2}
\end{figure}



