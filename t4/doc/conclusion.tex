\section{Conclusion}
\label{sec:conclusion}

\paragraph{}
In this fourth laboratory assignment, all the major goals of the project were achieved. We concluded with success a further interaction with a new software (Ubuntu), with a simulation software (Ngspice), with a computational language program (GNU Octave) and with a text report editor (LaTeX). The construction and analysis of the circuit was also finished with success through simulation and theoretical interpretation, which allowed a good comparative analysis between the differences presented in the behaviour of a simulated and theoretical electric circuit. 

In this laboratory assignment, we were asked to build and simulate an Audio Amplifier circuit. By creating and simulating this circuit, composed of two stages: a Gain Stage and an Output Stage, we can safely say it was succesfuly implemented, based on the analysis of the percentual relative errors presented on Section~\ref{error}. As we can see, the simulation results matched the theoretical results with little deviation. Besides, the comparative analysis of the graphics ploted by NGSpice and Octave and presented side by side proved the similitude of the results obtained throughout this assignment. 

Ultimately, we can finish this report by acknowledging the fact that this laboratory assignemnt proved to be really sucessful and very helpful in the sense that we were able to build an Audio Amplifier circuit which allowed us to get a better understanding of how this type of circuits work.

