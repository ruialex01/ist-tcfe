\section{Theoretical Introduction}
\label{sec:theoretical}

\paragraph{}
Electric power is transported on wires either as a direct current (DC) flowing in one direction at a non-oscillating constant voltage, or as an alternating current (AC) flowing backwards and forwards due to an oscillating voltage. AC is the dominant method of transporting power because it offers several advantages over DC, including lower distribution costs and simple way of converting between voltage levels thanks to the invention of the transformer. AC power that is sent at high voltage over long distances and then converted down to a lower voltage is a more efficient and safer source of power in homes. 

AC to DC Converters are one of the most important tools in power electronics because a lot of real applications are based on this type of conversions. The AC current to DC current conversion process is known as rectification. This rectifier converts the  AC supply into the DC supply at the  load end connection. Normally, transformers are used to adjust the AC source to get the step down transformer to reduce the voltage amplitude, so that there is a better operation range for the DC supply.

\paragraph{}
The output voltage of the full-wave rectifier is not constant, it is always oscillating and thus can’t be used in real-life applications. That's why it is required a DC supply with a constant output voltage. This need can be fulfilled by using an adequate filter with an inductor or a capacitor (envelope detector circuit) to make the output voltage smooth and constant.

This capacitor is connected in parallel to the load resistance in a linear power supply. The capacitor is used to increase the DC voltage and to reduce the ripple voltage of the output obtained. This capacitor is also called a reservoir or smoothing capacitor and it is generally followed by a voltage regulator which eliminates the remaining ripples so that the required output can be achieved.

While the rectifier conducts and the potential is higher than the charge across the capacitor, the capacitor stores the energy from the transformer. However,  when the output of the rectifier falls below the charge across the capacitor, the capacitor naturally discharges its energy into the circuit. As the rectifier conducts current only in the forward direction, all the energy discharged by the capacitor will flow into the load. There are other types of filters, such as the half wave rectifier, studied in the theoretical classes. However, the efficiency of the full-wave rectifier is double that of a half-wave and the ripple voltage is lower using this four diodes bridge rectifier.

\paragraph{}
Finnaly, a voltage regulator is a circuit that creates and maintains a fixed output voltage, regardless of changes to the input voltage or load conditions. A simple voltage regulator can be made from a resistor in series with a diode or a series of diodes. The voltage regulators keep the voltages from a power supply within a range that is compatible with the other electrical components.
